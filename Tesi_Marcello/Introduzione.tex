\chapter*{Ringraziamenti}
Desidero ringraziare innanzitutto il relatore di questa tesi, il dott. Francesco Pasquale per la disponibilit\`a, l'attenzione e la gentilezza dimostrate durante la stesura del lavoro, ma sopratutto per il forte interesse che mi ha suscitato negli argomenti trattati durante le sue lezioni.\\
Inoltre vorrei ringraziare la mia famiglia, il mio punto di riferimento. In particolare mio padre che ha sempre creduto in me e con i suoi sacrifici mi ha permesso di raggiungere questo traguardo. Mia madre, che mi ha spronato e sostenuto in ogni momento, aiutandomi a trovare la grinta anche nelle circostanze pi\`u difficili. Mia sorella che, forse inconsapevolmente, mi trasmette continuamente la gioia e la spensieratezza e mi da la certezza di essermi sempre vicino.\\
Ai miei compagni di studio e non, e a chi mi \`e stato sempre a fianco.\\
Un sentito grazie a tutti!


\chapter*{Introduzione}
Sempre un numero pi\`u alto di servizi internet basano la loro infrastruttura di rete su un'architettura peer-to-peer (P2P), la quale permette una notevole resilienza, flessibilit\`a e rapidit\`a di adattamento.
Una delle sfide maggiori in questo tipo di reti \`e di garantire robustezza a prescindere da malfunzionamenti o attacchi esterni. La conoscenza della topologia della rete pu\`o essere la chiave sia di un miglioramente della performance del servizio che si poggia su di essa, sia di un attacco pi\`u mirato alla rete stessa, come ad esempio un \textit{denial of service}.
Inoltre l'anonimia potrebbe essere un obiettivo (e.g. \textit{TOR}) che pu\`o venire compromessa se un hacker \`e in grado di ricostruire i \textit{path} di comunicazione tra i nodi.\\
Il sistema di moneta digitale Bitcoin utilizza una rete P2P per trasmettere informazioni riguardanti le transazioni attraverso i partecipanti della stessa. Lo scambio di informazione segue il \textit{gossip protocol}: se un \textit{peer} riceve una nuova transazione, controlla che questa sia valida e la inotlra ai suoi vicini.
A  partire  dalla  sua  creazione nel 2009, Bitcoin ha conquistato un'attenzione sempre maggiore, complice  anche  il \textit{boom} osservato nel 2017, anno nel quale la valuta ha raggiunto una quotazione pari a circa 3.000.
L'obiettivo principale di Bitcoin \`e quello di garantire trasferimenti di denaro sicuro che non si basino sulla fiducia in terze parti (come ad esempio una banca), e questo  obiettivo  viene  raggiunto  distribuendo  una  copia  del  database  delle transazioni su ogni nodo del network, divenendo cosi la prima forma di pagamento \textit{trustless}.
Il  database, che prende il nome di \textit{blockchain}, tiene traccia di tutte le  transazioni  avvenute  nella  rete  dal  1 gennaio 2009. 

La validit\`a delle transazioni viene garantita da un meccanismo a firma digitale tramite chiave privata e pubblica, permettendo che solamente i reali possessori di una determinata somma possano spenderla.  Le transazioni vengono verificate, prima di essere immesse nella blockchain, da nodi speciali, denominati \textit{miners}. I \textit{miners} svolgono il duplice ruolo di verifica della transazioni e di emissione dei bitcoin. La \textit{blockchain} non  viene  aggiornata  in  \textit{real time}, ma viene aggiunto un gruppo di transazioni, chiamato blocco, in media ogni 10 minuti . Ogni volta che tale blocco viene aggiunto, il sistema emette una quantit\`a prestabilita di bitcoin , destinata al minatore che per primo \`e riuscito a verificarne la validit\`a.  Queste operazioni richiedono un notevole potere computazionale, perci\`o i minatori sono molti meno dei nodi totali della rete, ma il loro numero \`e comuqnue sufficiente a garantire la decentralizzazione della stessa. Un'ulteriore caratteristica di Bitcoin \`e che l'emissione dell'omonima valuta, denominata di seguito BTC, \`e limitata, e cresce nel tempo asintoticamente fino a raggiungere il limite prestabilito di 21 milioni di unit\`a.  In questo modo il valore non pu\`o essere manipolato tramite l'inflazione, come invece pu\`o accadere con le valute tradizionali. L'introduzione fin qui fatta \`e  volutamente  generica, poich\`e i dettagli tecnici verranno analizzati in seguito.
Si \`e scelto di suddividere il lavoro di questa tesi in tre parti. Nel primo capitolo si vedr\`a una spiegazione generale di quello che \`e il mondo Bitcoin, come tale tecnologia \`e nata e si \`e diffusa, e come un utente comune pu\`o usufruirne. Nella secondo capitolo invece verr\`a fatta un analisi pi\`u tecnica sulle principali innovazioni che hanno permesso di creare tutta l'infrastruttura che sta alla base del Bitcoin, mentre negli ultimi due capitoli mostreremo come con un applicativo, creato appositamente per questo fine, siamo riusciti a ricavare informazioni utili da dati reali riguardanti le connessioni di un \textit{full node} con i suoi \textit{peer}.



\section*{Scopo del lavoro}
In questo lavoro mi concentrer\`o a studiare la topologia della rete Bitcoin, o almeno come questa viene osservata da parte di un singolo \textit{full node}. Attraverso degli applicativi sviluppati appositamente per questo scopo, andremo a raccogliere dati che in seguito verranno elaborati allo scopo di estrarre informazioni utili riguardo le connessioni di un nodo verso altri \textit{peer} del network Bitcoin. Il risultato di questo lavoro dovr\`a essere visto comunque come un passo intermedio per il raggiungimenro di un obiettivo pi\`u ambizioso che punta ad inferire e a studiare le connessioni dell'intero network Bitcoin.



